\documentclass[12pt]{article}

\usepackage{sbc-template}
\usepackage{graphicx,url}
\usepackage[utf8]{inputenc}
\usepackage[brazil]{babel}
%\usepackage[latin1]{inputenc}  

     
\sloppy

%% \title{Instructions for Authors of SBC Conferences\\ Papers and Abstracts}
%%\title{Modelo de Apoio ao Ensino em Ambientes Virtuais de Aprendizagem Sustentado por Consciência Situacional}
\title{Consciência Situacional em Ambientes Virtuais de Aprendizagem:
	Modelo Mental Proposto em Obras de Mineração de Dados Educacionais }

\author{Ernani Martins\inst{1}, Luciana Assis\inst{2}, Cláudia Berti\inst{2}, Alessandro Vivas\inst{2} }


\address{Instituto de Informática -- Universidade Federal do Rio Grande do Sul
  (UFRGS)\\
  Caixa Postal 15.064 -- 91.501-970 -- Porto Alegre -- RS -- Brazil
\nextinstitute
  Departamento de Computação \\ Universidade Federal dos Vales do Jequitinhonha e Mucuri (UFVJM) \\
  Diamantina, MG -- Brazil
  \email{\{naninmartins,lupassis,claudiabberti,alessandro.vivas\}@gmail.com}
}

\begin{document} 

\maketitle

\begin{abstract}
  This article describes the use of Situational Awareness (CS) in support of Virtual Learning Environments (AVA's). Educational Data Mining (EDM) researches the development of methods for unique types of data generated from educational environments, however AVA's tend to result in extremely dynamic datasets, thus requiring tools that adapt to the variations arising from each interaction with the user. The CS uses Mental Models to systematically assimilate states in certain situations, the conscious view about the environment allows the identification of each configuration of the data and what better action to take, being able to delimit situations in which the MDE methods best apply.
\end{abstract}
     
\begin{resumo} 
  Este artigo descreve o uso da Consciência Situacional (CS) no suporte à Ambientes Virtuais de Aprendizagem (AVA's). Mineração de Dados Educacionais (MDE) buscam o desenvolvimento de métodos para tipos únicos de dados gerados a partir de ambientes educacionais, no entanto AVA's tendem a resultar em conjuntos de dados extremamente dinâmicos, requerendo assim ferramentas que se adaptem as variações decorrentes de cada interação com o usuário. A CS usa Modelos Mentais para assimilar sistematicamente estados em determinadas situações, a visão consciente sobre o ambiente permite a identificação de cada configuração dos dados e qual melhor ação a ser tomada podendo delimitar situações nos quais os métodos da MDE melhor se aplicam.
\end{resumo}


\section{Introdução} 

O recente aprimoramento da tecnologia tem fornecido grande flexibilidade aos educadores para disseminar o conhecimento nas mais inúmeras plataformas. \cite{Ahmad_Shamsuddin_2010} afirmam que o uso destas tecnologias disponibilizam diversas abordagens para facilitar o ensino, maximizando os resultados de aprendizagem entre os discentes.

Ambientes Virtuais de Aprendizagem (AVA's) criam modelagens e diretrizes que possam inferir o estado do aprendizado de cada estudante, assim modalidades de ensino da Educação a Distância caracterizam-se por práticas pedagógicas personalizadas utilizando-se das mais diversas tecnologias da informação em vários graus educacionais. 

A utilização das plataformas pelos discentes resulta em um conjunto diverso de dados em inúmeras situações de interação com o ambiente, este valor numeroso de dados pode dificultar o gerenciamento e análise sobre a relação dos agentes junto dos AVA's \cite{Rabelo_et_al2017}. Alunos e professores relacionam-se por meio da disponibilização de materiais, discussão em fóruns sobre determinados assuntos e chats, todavia, estes meios por vezes não são o bastante para que os discentes consigam atingir o conhecimento em sua melhor forma \cite{Falci_et_al_2018}. 

\cite{Endsley2012} enfatizam que o grande volume de dados em sua maioria não permitem extrair um conhecimento sucinto e real da situação. A dinamicidade dos acontecimentos no ambiente deverá ser capturada e assimilada em um processo ininterrupto, integrando a percepção dos elementos no espaço, assim como a compreensão e projeção de eventos imediatamente no futuro \cite{Silva_et_al_2012}.

O entendimento de aspectos existentes no ambiente é chamado de Consciência Situacional-CS (do inglês Situation Awareness), \cite[p.13]{Endsley2012} compreendem a CS em `` estar ciente do que está acontecendo ao seu redor e entendendo o que estas informações significam agora e no futuro ". 

\cite{Roy_Breton_Rousseau_2007} afirmam que a CS é um componente natural da organização cognitiva humana, e os benefícios que resultam de um melhor entendimento da situação podem ser percebidos desde a pré-história. Uma Decisão assertiva torna-se difícil quando não existe uma boa consciência da situação \cite{Endsley2012}. 

Técnicas e tecnologias baseadas em Mineração de Dados Educacionais (MDE) e a CS podem facilitar o processo de Tomada de Decisão e entendimento do ambiente, possibilitando assim a construção de sistemas com conteúdo adaptativo ao processo de aprendizagem do aluno. A MDE emprega diversas técnicas e procedimentos sobre uma base de dados educacional desejando a descoberta de conhecimento relevante. 

\cite{Endsley1995} e \cite{Endsley2012}  relatam que a CS emprega Modelos Mentais na criação de estruturas complexas  para mapear o comportamento de sistemas específicos diminuindo assim a carga de dados sobre um usuário. Este arcabouço de situações sugere qual procedimento a ser tomado dada determinada situação. 

Modelos Mentais podem auxiliar na escolha dos métodos de MDE sobre determinadas circunstâncias, o resultado do processamento destes métodos influencia diretamente no nível de CS que o ambiente educacional atingirá. 

\section{Fundamentação Teórica} \label{sec:firstpage}

BORA BORA FIII

\section{Abordagem Proposta}

In some conferences, the papers are published on CD-ROM while only the
abstract is published in the printed Proceedings. In this case, authors are
invited to prepare two final versions of the paper. One, complete, to be
published on the CD and the other, containing only the first page, with
abstract and ``resumo'' (for papers in Portuguese).

\section{Sections and Paragraphs}

Section titles must be in boldface, 13pt, flush left. There should be an extra
12 pt of space before each title. Section numbering is optional. The first
paragraph of each section should not be indented, while the first lines of
subsequent paragraphs should be indented by 1.27 cm.

\subsection{Subsections}

The subsection titles must be in boldface, 12pt, flush left.



\section{Conclusão e Trabalhos Futuros}
Então Aqui vão as conclusões e trabalhos futuros do que vai rolar nessa bagaça tretistica

\bibliographystyle{sbc}
\bibliography{library}

\end{document}
