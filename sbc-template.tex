\documentclass[12pt]{article}

\usepackage{sbc-template}
\usepackage{graphicx,url}
\usepackage[utf8]{inputenc}
\usepackage[brazil]{babel}
%\usepackage[latin1]{inputenc}  

     
\sloppy

%% \title{Instructions for Authors of SBC Conferences\\ Papers and Abstracts}
%%\title{Modelo de Apoio ao Ensino em Ambientes Virtuais de Aprendizagem Sustentado por Consciência Situacional}
\title{Consciência Situacional em Ambientes Virtuais de Aprendizagem:
	Modelo Mental Proposto em Obras de Mineração de Dados Educacionais }

\author{Ernani Martins\inst{1}, Luciana Assis\inst{2}, Cláudia Berti\inst{2}, Alessandro Vivas\inst{2} }


\address{Instituto de Informática -- Universidade Federal do Rio Grande do Sul
  (UFRGS)\\
  Caixa Postal 15.064 -- 91.501-970 -- Porto Alegre -- RS -- Brazil
\nextinstitute
  Departamento de Computação \\ Universidade Federal dos Vales do Jequitinhonha e Mucuri (UFVJM) \\
  Diamantina, MG -- Brazil
  \email{\{naninmartins,lupassis,claudiabberti,alessandro.vivas\}@gmail.com}
}

\begin{document} 

\maketitle

\begin{abstract}
  This article describes the use of Situational Awareness (CS) in support of Virtual Learning Environments (AVA's). Educational Data Mining (EDM) researches the development of methods for unique types of data generated from educational environments, however AVA's tend to result in extremely dynamic datasets, thus requiring tools that adapt to the variations arising from each interaction with the user. The CS uses Mental Models to systematically assimilate states in certain situations, the conscious view about the environment allows the identification of each configuration of the data and what better action to take, being able to delimit situations in which the MDE methods best apply.
\end{abstract}
     
\begin{resumo} 
  Este artigo descreve o uso da Consciência Situacional (CS) no suporte à Ambientes Virtuais de Aprendizagem (AVA's). Mineração de Dados Educacionais (MDE) buscam o desenvolvimento de métodos para tipos únicos de dados gerados a partir de ambientes educacionais, no entanto AVA's tendem a resultar em conjuntos de dados extremamente dinâmicos, requerendo assim ferramentas que se adaptem as variações decorrentes de cada interação com o usuário. A CS usa Modelos Mentais para assimilar sistematicamente estados em determinadas situações, a visão consciente sobre o ambiente permite a identificação de cada configuração dos dados e qual melhor ação a ser tomada podendo delimitar situações nos quais os métodos da MDE melhor se aplicam.
\end{resumo}


\section{Introdução} 

O recente aprimoramento da tecnologia tem fornecido grande flexibilidade aos educadores para disseminar o conhecimento nas mais inúmeras plataformas. \cite{Ahmad_Shamsuddin_2010} afirmam que o uso destas tecnologias disponibilizam diversas abordagens para facilitar o ensino, maximizando os resultados de aprendizagem entre os discentes.

Ambientes Virtuais de Aprendizagem (AVA's) criam modelagens e diretrizes que possam inferir o estado do aprendizado de cada estudante, assim modalidades de ensino da Educação a Distância caracterizam-se por práticas pedagógicas personalizadas utilizando-se das mais diversas tecnologias da informação em vários graus educacionais. 

A utilização das plataformas pelos discentes resulta em um conjunto diverso de dados em inúmeras situações de interação com o ambiente, este valor numeroso de dados pode dificultar o gerenciamento e análise sobre a relação dos agentes junto dos AVA's \cite{Rabelo_et_al2017}. Alunos e professores relacionam-se por meio da disponibilização de materiais, discussão em fóruns sobre determinados assuntos e chats, todavia, estes meios por vezes não são o bastante para que os discentes consigam atingir o conhecimento em sua melhor forma \cite{Falci_et_al_2018}. 

\cite{Endsley2012} enfatizam que o grande volume de dados em sua maioria não permitem extrair um conhecimento sucinto e real da situação. A dinamicidade dos acontecimentos no ambiente deverá ser capturada e assimilada em um processo ininterrupto, integrando a percepção dos elementos no espaço, assim como a compreensão e projeção de eventos imediatamente no futuro \cite{Silva_et_al_2012}.

O entendimento de aspectos existentes no ambiente é chamado de Consciência Situacional-CS (do inglês Situation Awareness), \cite[p.13]{Endsley2012} compreendem a CS em `` estar ciente do que está acontecendo ao seu redor e entendendo o que estas informações significam agora e no futuro ". 

\cite{Roy_Breton_Rousseau_2007} afirmam que a CS é um componente natural da organização cognitiva humana, e os benefícios que resultam de um melhor entendimento da situação podem ser percebidos desde a pré-história. Uma Decisão assertiva torna-se difícil quando não existe uma boa consciência da situação \cite{Endsley2012}. 

Técnicas e tecnologias baseadas em Mineração de Dados Educacionais (MDE) e a CS podem facilitar o processo de Tomada de Decisão e entendimento do ambiente, possibilitando assim a construção de sistemas com conteúdo adaptativo ao processo de aprendizagem do aluno. A MDE emprega diversas técnicas e procedimentos sobre uma base de dados educacional desejando a descoberta de conhecimento relevante. 

\cite{Endsley1995} e \cite{Endsley2012}  relatam que a CS emprega Modelos Mentais na criação de estruturas complexas  para mapear o comportamento de sistemas específicos diminuindo assim a carga de dados sobre um usuário. Este arcabouço de situações sugere qual procedimento a ser tomado dada determinada situação. 

Modelos Mentais podem auxiliar na escolha dos métodos de MDE sobre determinadas circunstâncias, o resultado do processamento destes métodos influencia diretamente no nível de CS que o ambiente educacional atingirá. 

\section{Fundamentação Teórica} \label{sec:firstpage}
	
Este capítulo provê uma base teórica referente a formalização dos conceitos de Consciência Situacional e Mineração de Dados voltado a educação, em razão de facilitar o processo de tomada de decisão em ambientes educacionais.  

\subsection{Consciência Situacional}

Define-se CS como: `` a percepção dos elementos no ambiente dentro de um volume de tempo e espaço, a compreensão dos seus significados, e a projeção dos seus estados em um futuro próximo " \cite [p. 97]{Endsley1988}.

A falta de CS leva as pessoas a um desentendimento da situação em que se encontram, sendo que a melhor maneira de auxiliar a avaliação humana é suprindo o usuário com altos níveis de Consciência Situacional \cite{Endsley2012}. Todas as tarefas cotidianas requerem níveis específicos de consciência, ao cozinhar por exemplo, a pessoa deve entender todos os parâmetros do ambiente, como temperatura da panela, quantidade de água esquentando, previsão de fervura da água e etc.

Existem algumas variações nos modelos de definição da consciência da situação, contudo a definição melhor aceita na literatura para modelagem computacional foi proposta por \cite{Endsley1995} sendo separados em 3 níveis: \emph{percepção, compreensão e projeção}.

\begin{itemize}
		
	\item \textbf{\textit{Percepção} dos elementos do ambiente:}
	
	Neste estado é necessário perceber os sinais do ambiente, variáveis relevantes, elementos e atributos do ambiente, é o primeiro passo para obtenção da Consciência Situacional. Sem uma boa percepção, informações relevantes as etapas de compreensão e projeção ficam incompletas, levando a interpretações ruidosas e baixa consciência sobre os estados do ambiente. A percepção destas informações são captadas por sensores ou pela combinação deles (ex: olfato e paladar em humanos, lasers e radares em máquinas).
	
	\item  \textbf{\textit{Compreensão} da situação atual:}
	
	O segundo passo na obtenção da CS é conseguir realizar o entendimento mais correto possível sobre os dados colhidos, compreendendo as relações e dinamismos percebidas em sinais que serão relevantes para alcançar os objetivos. Baseado em conhecimentos dos elementos do nível 1, o tomador de decisão reconhece padrões e formas em uma figura holística do ambiente, compreendendo a significância dos objetos e eventos.	
	
	\item \textbf{\textit{Projeção} do estado futuro:}
	
	Após a compreensão dos fatos, nesta etapa deve-se ser capaz de predizer um estado futuro, ou seja, a habilidade de antever eventos onde tomadores de decisão necessitam de um alto nível de SA. O Nível 3 somente é obtido a partir de uma boa compreensão (Nível 2). \cite{Endsley2012} exemplificam que: `` com Nível 3 de CS, uma motorista sabe que se prosseguir para dentro do cruzamento da rua, ela provavelmente será atingida. "
	
\end{itemize}

\subsection{Modelos Mentais}

Para \cite{Endsley2012} o indivíduo  possui dois tipos de memórias: \textit{de curto prazo ou de trabalho e de longo prazo}. Quando armazenamos informações em \textit{memória de trabalho}, gravamos o conhecimento em uma base temporária na mente, entretanto somente uma quantia restrita de informação consegue ser retida e manipulada, sendo que uma pessoa deverá relacionar-se ativamente com estas informações para não esquecê-las. A informação conciliada com conhecimento prévio em uma memória de trabalho cria uma nova imagem mental ou a atualiza conforme mudanças na situação. 	

Imagem mental são formas de representar em pensamento o mundo externo, a mente humana capta o mundo exterior a partir de representações mentais. Estas visões podem ser categorizadas entre representações \textit{analógicas e proposicionais}, a imagem visual é um exemplo de representação analógica \cite{Moreira1996}. 

Por outro lado as representações proposicionais são abstratas, organizadas em regras rígidas compreendendo o conteúdo ideacional da mente. Estas representações proposicionais mapeiam-se em uma linguagem da mente (`` mentalês "), de forma que representações proposicionais não são pensamentos expressos em frases, mas sim entidades individuais e abstratas formuladas em linguagem própria da mente, alguns psicólogos cognitivos afirmam que a imagem mental pode ser reduzida a representações proposicionais  \cite{Moreira1996}.

Memórias de longo-prazo estruturadas podem ser utilizadas para contornar as limitações em memórias de trabalho, gerenciando o conhecimento em modelos mentais, esquemas e scripts, desempenhando uma importante função na CS \cite{Endsley1995}. 

/% Modelos Mentais foram definidos por \apud [p.60]{Rouse1985}{Endsley1995} como: `` mecanismos pelos quais humanos são capazes de gerar descrições e formas de sistemas propostos, explicações de funcionalidades e estados observados do sistema, e previsões de estados futuros ".
%/
\cite{Endsley1995} e \cite{Endsley2012} entendem que um Modelo Mental são estruturas complexas para entender o comportamento de sistemas específicos, ou seja, uma assimilação sistemática do funcionamento de algo. 

/%Um modelo mental ajuda pessoas a perceberem quais informações são mais importantes, \cite{Berti2017} ilustra que um pedestre sabe que é mais importante olhar para os dois lados da via antes de atravessá-la do que olhar para o céu. Esta inferência se dá pois no devido momento de atravessar a rua existe um acesso dos modelos mentais conjuntos com experiências e entradas do ambiente, afirmando ao pedestre que olhar para os dois lados é uma informação crucial a ser entendida. A existência de um bom modelo mental é fator crucial para garantir altos níveis de CS 2 e 3.%/

Um \textit{esquema} é um estado provável em que podemos acessar a partir de um modelo mental, a mente define padrões de acontecimentos anteriores e os reconhecessem conforme as entradas do ambiente \cite{Endsley2012}. 

Os esquemas podem funcionar como atalhos para que não necessitemos acessar o modelo mental a todo momento, (como se tivéssemos uma sensação de \textit{dejavú}). Estes esquemas são formados a partir de casos vividos, contudo, uma vantagem dos esquemas são que não necessitam representar exatamente outras situações parecidas pois as pessoas tem a capacidade de relacionar características de uma situação com um esquema. Um médico relaciona um esquema a partir da observação dos sintomas de um paciente podendo assim deduzir um estado do esquema e uma doença \cite{Endsley2012}.

Um \textit{script} são sequências de ações sendo associadas a um esquema,ou seja, são passos do que se deve fazer a partir da seleção de uma ação, os scripts também são desenvolvidos a partir da experiência, ou são normatizados pelo domínio. Novamente um doutor ao selecionar um esquema (determinando um estado do paciente), prosseguirá o tratamento dando continuidade a uma série de ações pré-programadas para determinada situação \cite{Endsley2012}.  

\cite{Moreira1996} acredita que não existe um único modelo mental para determinadas situações ou estados em uma mesma abordagem dos fatos, ainda que um modelo mental demonstre-se mais adequado para representação da situação. Um mecânico pode seguir trocando entre vários modelos mentais e esquemas conforme segue na identificação das entradas (possíveis problemas) do seu exame no carro.

\subsection{Mineração de Dados Educacionais}

A Mineração de Dados foi desenvolvida com o intuito de permitir descoberta de conhecimento sobre uma base de dados, \cite{Goldschmidt_Passos_2005} afirmam que este conjunto de técnicas oriundas da Estatística e Inteligência Artificial visam obter conhecimento novo, útil, relevante e não-trivial os quais possam estar escondidos em tais bases.

\cite{Leite_et_al_2016} reforçam que o uso de técnicas de Mineração de dados ao contexto educacional ou MDE (Mineração de Dados Educacionais) são soluções promissoras para a compreensão de informações nas base de dados em AVA's.

\cite{Romero_Ventura_2013} concordam em dizer que MDE pode ser definida como a aplicação de técnicas de MD para o específico tipo de conjunto de dados originados de ambientes educacionais, combinando ciência da computação, estatística e educação .

\cite{Garcia_et_al_2011} e \cite{Santos2016} elucidam o processo de MDE em uma conversão de dados brutos de Sistemas Educacionais em informação, que podem ser usadas por desenvolvedores de software, professores, pesquisadores educacionais, em informação útil, \cite{Garcia_et_al_2011} ainda afirmam que o processo de mineração de dados educacionais é baseado nos mesmos passos de um processo de MD como exposto por \cite{Romero_Ventura_Bra_2004} nas seguintes etapas:

\begin{itemize}
	
	\item Pré-processamento: Os dados obtidos a partir de ambientes educacionais devem ser pré-processados, transformando-os em formatos apropriados para a mineração.
	\item Mineração de Dados: Passo central do procedimento onde as técnicas de MD são aplicadas (ex:regressão, classificação, associação etc..).
	\item Pós-processamento: Os resultados ou modelos obtidos são interpretados e usados no processo de tomada de decisão sobre o ambiente educacional.
	
\end{itemize}

\section{Consciência Situacional e Data Mining}

Um software com CS deve relacionar todas informações contextuais disponíveis procurando obter o máximo de entendimento sobre o ambiente, assim estas informações devem ser organizadas em modelos mentais em determinadas situações prototípicas, ou seja, cada relação entre os atores da situação devem ser vistas a partir de um modelo da situação \cite{Berti2017}. 

A mineração de dados refere-se a descoberta de conhecimento e análise sobre as bases de informações e sobre os modelos que são definidos no software, \cite{Krishnaswamy_et_al_2005}	relacionam uma situação com o estado do relacionamento de uma entidade. Em seu trabalho buscando a CS para aumentar a segurança em rodovias, eles definiram os papéis da consciência do ambiente em meios de modelar as informações contextuais sobre o motorista, veículo e ambiente no qual o carro está situado, e posteriormente utilizar as técnicas de mineração para uma análise sobre modelos pré-definidos.

\cite{Mitsch_et_al_2013} expõem que em sistemas com CS, definir situações críticas requer uma quantidade significativa de tempo e esforço. Este esforço de disponibilizar conhecimento explícito poderia ser facilitado e complementado através de MD, descobrindo relações "interessantes" ou incomuns em domínios nos quais os especialistas podem não estar explicitamente conscientes (conhecimento intrínseco). Em tempo de execução de um sistema com CS, novas e contínuas mudanças e relações incomuns podem surgir, as quais não foram nitidamente observadas ou definidas. O autor ainda ressalta a grande diversidade dos dados de entrada, monitorando vários objetos inter-relacionados em espaço e tempo heterogêneos que devem ser tratados na avaliação de uma situação crítica.

\cite{Berti2017} considera o modelo de Endsley intuitivo e claro, permitindo aos estudiosos mensurar em um caminho simples as composições e os requisitos da consciência em cada um dos três níveis.  

/% \apud{McGuinnes_Foy_2000}{Salerno_Hinman_Boulware_2004} estenderam o modelo de Endsley adicionando um quarto nível nomeado Resolução, este nível viabilizaria consciência do melhor caminho a seguir para obter a definição desejada da situação, assim durante o fluxo do modelo aplicaria-se os seguintes questionamentos por etapa:
%/
\begin{itemize}
	
	\item Percepção: `` Quais são os fatos atuais? " 
	\item Compreensão: `` O que está atualmente acontecendo? "
	\item Projeção: `` O que geralmente acontece se...? "
	\item Resolução: `` O que exatamente eu deveria fazer? "
	
\end{itemize}

Tais indagações norteiam à compreensão da situação alicerçado em conhecimentos de situações similares ocorridas no passado e no presente, se um conhecimento prévio não existe, é preciso aprende-lo ou descobri-lo, papel este na maioria das vezes destinado principalmente a técnicas de Mineração de Dados aliadas a outras ferramentas de Descoberta do Conhecimento \cite{Salerno_Hinman_Boulware_2004}. 

\section{Considerações Finais}

Em AVA's podemos ter inúmeras ações e situações de interação de um aluno com uma máquina, em casos que podem levar a diversas interpretações, assim como inúmeros tipos de informações de entrada no ambiente (texto, áudio, frequência de acesso etc), conhecer e identificar estas interpretações é desafiador, pois cada reconhecimento de uma situação pode acarretar inúmeras respostas diferentes. 

Um sistema com CS deve trabalhar continuamente para criar e atualizar modelos de situações nos quais podemos trabalhar tanto com técnicas de mineração como regras de inferência, lógica etc. Cada tipo de entrada no sistema pode demonstrar uma resposta mais efetiva para determinadas técnicas e modelos, um sistema pode ter uma resposta melhor para determinado mapa mental dada uma situação. 

É recorrente na literatura comparativos sobre performances de algoritmos em AVA e e-Learning, entretanto, não é recorrente na área estudos que proponham arquiteturas visando atingir um nível de consciência situacional em situações nas quais tais técnicas e procedimentos apresentam uma melhor performance.  






\section{Abordagem Proposta}

Section titles must be in boldface, 13pt, flush left. There should be an extra
12 pt of space before each title. Section numbering is optional. The first
paragraph of each section should not be indented, while the first lines of
subsequent paragraphs should be indented by 1.27 cm.

\subsection{Subsections}

The subsection titles must be in boldface, 12pt, flush left.



\section{Conclusão e Trabalhos Futuros}
Então Aqui vão as conclusões e trabalhos futuros do que vai rolar nessa bagaça tretistica

\bibliographystyle{sbc}
\bibliography{library}

\end{document}
